\input{../utils/preamble}
\createdgmtitle{6}
%--------------------------------------------------------------------------------
\begin{document}
%--------------------------------------------------------------------------------
\begin{frame}[noframenumbering,plain]
%\thispagestyle{empty}
\titlepage
\end{frame}
%=======
\begin{frame}{Recap of previous lecture}
	\begin{block}{Change of variable theorem (CoV)}
		Let $\bx$ be a random variable with density function $p(\bx)$ and $f: \mathbb{R}^m \rightarrow \mathbb{R}^m$ is a differentiable, invertible function (diffeomorphism). If $\bz = f(\bx)$, $\bx = f^{-1}(\bz) = g(\bz)$, then
		\begin{align*}
			p(\bx) &= p(\bz) |\det(\bJ_f)| = p(\bz) \left|\det \left(  \frac{\partial \bz}{\partial \bx} \right) \right| = p(f(\bx)) \left|\det \left(  \frac{\partial f(\bx)}{\partial \bx} \right) \right| \\
			p(\bz) &= p(\bx) |\det(\bJ_g)|= p(\bx) \left|\det \left(  \frac{\partial \bx}{\partial \bz} \right) \right| = p(g(\bz)) \left|\det \left(  \frac{\partial g(\bz)}{\partial \bz} \right) \right|.
		\end{align*}
		\vspace{-0.5cm}
	\end{block}
	\begin{block}{Inverse function theorem}
		If function $f$ is invertible and Jacobian is continuous and non-singular, then
		\vspace{-0.3cm}
		\[
			\bJ_f = \bJ_{g^{-1}} = \bJ_g^{-1}, \quad |\det (\bJ_f)| = \frac{1}{|\det (\bJ_g)|}
		\]
		\vspace{-0.3cm}
	\end{block}
\end{frame}
%=======
\begin{frame}{Recap of previous lecture}
	\begin{block}{MLE problem}
		\vspace{-0.3cm}
		\[
			p(\bx|\btheta) = p(\bz) \left|\det \left(  \frac{\partial \bz}{\partial \bx} \right) \right|  = p(f(\bx, \btheta)) \left|\det \left( \frac{\partial f(\bx, \btheta)}{\partial \bx} \right) \right|
		\]
		\[
			\log p(\bx|\btheta) = \log p(f(\bx, \btheta)) + \log  |\det (\bJ_f) | \rightarrow \max_{\btheta}
		\]
	\end{block}
	\vspace{-0.2cm}
	\begin{figure}
		\includegraphics[width=0.85\linewidth]{figs/flows_how2}
	\end{figure}
	\myfootnotewithlink{https://arxiv.org/abs/1605.08803}{Dinh L., Sohl-Dickstein J., Bengio S. Density estimation using Real NVP, 2016} 
\end{frame}
%=======
\begin{frame}{Recap of previous lecture}
	\[
		\log p(\bx|\btheta) = \log p(f(\bx, \btheta)) + \log  |\det (\bJ_f) | \rightarrow \max_{\btheta}
	\]
	\vspace{-0.3cm}
	\begin{block}{Definition}
		Normalizing flow is a \textit{differentiable, invertible} mapping from data $\bx$ to the noise $\bz$. 
	\end{block}
	\begin{itemize}
		\item \textbf{Normalizing} means that the inverse flow takes samples from $p(\bx)$ and normalizes them into samples from density $p(\bz)$.
		\item \textbf{Flow} refers to the trajectory followed by samples from $p(\bz)$ as they are transformed by the sequence of transformations
		\[
			\bz = f_K \circ \dots \circ f_1(\bx); \quad \bx = f_1^{-1} \circ \dots \circ f_K^{-1} (\bz) = g_1 \circ \dots \circ g_K(\bz) 
		\] 
		\vspace{-0.4cm}
		\begin{block}{Log likelihood}
			\vspace{-0.4cm}
			\[
				\log p(\bx | \btheta) = \log p(f_K \circ \dots \circ f_1(\bx)) + \sum_{k=1}^K\log |\det (\bJ_{f_k})|,
			\]
			\vspace{-0.4cm} \\
			where $\bJ_{f_k} = \frac{\partial \mathbf{f}_k}{\partial \mathbf{f}_{k-1}}$.
		\end{block}
	\end{itemize}
\end{frame}
%=======
\begin{frame}{Recap of previous lecture}
	\begin{block}{Forward KL for flow model}
	  	\vspace{-0.1cm}
		\[
			\log p(\bx|\btheta) = \log p(f(\bx, \btheta)) + \log  |\det (\bJ_f)|
		\]
		\vspace{-0.3cm}
	\end{block}
	\begin{block}{Reverse KL for flow model}
  		\vspace{-0.5cm}
		\[
			KL(p || \pi)  = \bbE_{p(\bz)} \left[  \log p(\bz) -  \log |\det (\bJ_g)| - \log \pi(g(\bz, \btheta)) \right]
		\]
		\vspace{-0.5cm}
	\end{block}
	\begin{block}{Flow KL duality}
	  	\vspace{-0.3cm}
		\[
			\argmin_{\btheta} KL(\pi(\bx) || p(\bx | \btheta)) = \argmin_{\btheta} KL(p(\bz | \btheta) || p(\bz)).
		\]
		\vspace{-0.3cm}
		\begin{itemize}
			\item $p(\bz)$ is a base distribution; $\pi(\bx)$ is a data distribution;
			\item $\bz \sim p(\bz)$, $\bx = g(\bz, \btheta)$, $\bx \sim p(\bx| \btheta)$;
			\item $\bx \sim \pi(\bx)$, $\bz = f(\bx, \btheta)$, $\bz \sim p(\bz | \btheta)$;
		\end{itemize}
	\end{block}
	\myfootnotewithlink{https://arxiv.org/abs/1912.02762}{Papamakarios G. et al. Normalizing flows for probabilistic modeling and inference, 2019} 
\end{frame}
%=======
\begin{frame}{Recap of previous lecture}
	\vspace{-0.5cm}
	\begin{block}{Flow log-likelihood}
		\vspace{-0.3cm}
		\[
			\log p(\bx|\btheta) = \log p(f(\bx, \btheta)) + \log  |\det (\bJ_f)|
		\]
		\vspace{-0.5cm}
	\end{block}
	The main challenge is a determinant of the Jacobian.
	\begin{block}{Residual flows: planar/Sylvester}
		\vspace{-0.7cm}
		\[
			g(\bz, \btheta) = \bz + \mathbf{u} \, \sigma(\bw^T\bz + b); \quad 
			g(\bz, \btheta) = \bz + \bA \, \sigma(\bB\bz + \mathbf{b}).
		\]
		Matrix determinant lemma for calculating the Jacobian.
	\end{block}
	\begin{block}{Linear flows}	
		\vspace{-0.2cm}
		\[
			\bz = f(\bx, \btheta) = \bW \bx, \quad \bW \in \bbR^{m \times m}, \quad \btheta = \bW, \quad \bJ_f = \bW
		\]
		Matrix decompositions (LU or QR helps to parametrize matrix $\bW$ and reduce the cost of computing the $det(\bJ)$).
	\end{block}
	\myfootnote{\href{https://arxiv.org/abs/1505.05770}{Rezende D. J., Mohamed S. Variational Inference with Normalizing Flows, 2015}\\
	\href{https://arxiv.org/abs/1803.05649}{Berg R. et al. Sylvester normalizing flows for variational inference, 2018} \\
	\href{https://arxiv.org/abs/1807.03039}{Kingma D. P., Dhariwal P. Glow: Generative Flow with Invertible 1x1 Convolutions, 2018}}
\end{frame}
%=======
\begin{frame}{Outline}
	\tableofcontents
\end{frame}
%=======
\section{Autoregressive flows}
%=======
\begin{frame}{Gaussian autoregressive model}
	Consider an autoregressive model
	\vspace{-0.3cm}
	{\small
	\[
		p(\bx | \btheta) = \prod_{j=1}^m p(x_j | \bx_{1:j - 1}, \btheta), \quad
		p(x_j | \bx_{1:j - 1}, \btheta) = \mathcal{N} \left(\mu_j(\bx_{1:j-1}), \sigma^2_j (\bx_{1:j-1})\right).
	\]
	}
	\vspace{-0.5cm}
	\begin{block}{Sampling: reparametrization trick}
		\[
			x_j = \sigma_j (\bx_{1:j-1}) \cdot z_j + \mu_j(\bx_{1:j-1}), \quad z_j \sim \mathcal{N}(0, 1).
		\]
	\end{block}
	\begin{block}{Inverse transform}
		\vspace{-0.3cm}
		\[
			z_j = \left(x_j - \mu_j(\bx_{1:j-1}) \right) \cdot \frac{1}{\sigma_j (\bx_{1:j-1}) }.
		\]
		\vspace{-0.3cm}
	\end{block}
	We have got an invertible and differentiable transform (it is an autoregressive flow with base distribution $\bz = \cN(0, 1)$!).
	\myfootnotewithlink{https://arxiv.org/abs/1606.04934}{Kingma D. P. et al. Improving Variational Inference with Inverse Autoregressive Flow, 2016} 
\end{frame}
%=======
\begin{frame}{Gaussian autoregressive flow}
	\vspace{-0.2cm}
	\begin{align*}
		\bx &= g(\bz, \btheta) \quad \Rightarrow \quad x_j = \sigma_j (\bx_{1:j-1}) \cdot z_j + \mu_j(\bx_{1:j-1}). \\
		\bz &= f(\bx, \btheta) \quad \Rightarrow \quad z_j = \left(x_j - \mu_j(\bx_{1:j-1}) \right) \cdot \frac{1}{ \sigma_j (\bx_{1:j-1})}.
	\end{align*}
	Generation function $g(\bz, \btheta)$ is \textbf{sequential}. Inference function $f(\bx, \btheta)$ is \textbf{not sequential}.
	\begin{block}{Forward KL for flow model}
		\vspace{-0.2cm}
		\[
		\log p(\bx|\btheta) = \log p(f(\bx, \btheta)) + \log  \left|\det \left( \frac{\partial f(\bx, \btheta)}{\partial \bx} \right) \right|
		\]
		\vspace{-0.2cm}
		\begin{itemize}
			\item We need to be able to compute $f(\bx, \btheta)$ and its Jacobian.
			\item We need to be able to compute the density $p(\bz)$.
			\item We don’t need to think about computing the function $g(\bz, \btheta) = f^{-1}(\bz, \btheta)$ until we want to sample from the flow.
		\end{itemize}
	\end{block}
	\myfootnotewithlink{https://arxiv.org/abs/1705.07057}{Papamakarios G., Pavlakou T., Murray I. Masked Autoregressive Flow for Density Estimation, 2017} 
\end{frame}
%=======
\begin{frame}{Masked autoregressive flow (MAF)}
	\begin{block}{Gaussian autoregressive model}
		\vspace{-0.7cm}
		\[
			p(\bx | \btheta) = \prod_{j=1}^m p(x_j | \bx_{1:j - 1}, \btheta) = \prod_{j=1}^m \mathcal{N} \left(x_j | \mu_j(\bx_{1:j-1}), \sigma^2_j (\bx_{1:j-1})\right).
		\]
		\vspace{-0.5cm}
	\end{block}
	We could use MADE for the conditionals. 
	Samples from the base distribution could be an indicator of how good the flow was fitted. 
	\vspace{-0.3cm}
	\begin{figure}
		\includegraphics[width=0.95\linewidth]{figs/maf1.png}
	\end{figure}
	MAF is just a stacked MADE model with different ordering.
	\begin{itemize}
		\item Parallel density estimation.
		\item Sequential sampling.
	\end{itemize}
	\myfootnotewithlink{https://arxiv.org/abs/1705.07057}{Papamakarios G., Pavlakou T., Murray I. Masked Autoregressive Flow for Density Estimation, 2017} 
\end{frame}
%=======
\begin{frame}{Autoregressive flows}
	\vspace{-0.3cm}
	\[
		x_j = \tau (z_j, c(\bz_{1:j-1})) \quad \Leftrightarrow \quad z_j = \tau^{-1} (x_j, c(\bz_{1:j-1}))
	\]
	\vspace{-0.3cm}
	\begin{itemize}
		\item $\tau (\cdot, \cdot)$ -- coupling law (invertible by first argument, differentiable).
		\item $c(\cdot)$ -- coupling function (do not need to be invertible, could be neural network).
	\end{itemize}
	\begin{block}{Coupling law $\tau(\cdot, \cdot)$}
		\begin{itemize}
			\item $\tau(x, c) = x + c$ -- additive;
			\item $\tau(x, c) = x \odot c_1 + c_2$ -- affine.
		\end{itemize}
	\end{block}
	What is the Jacobian for the additive/affine coupling law? 
	\begin{block}{Jacobian}
		\vspace{-0.3cm}
		\[
			\det \left( \frac{\partial \bx}{\partial \bz} \right) = \prod_{j=1}^m \frac{\partial x_j}{\partial z_j} = \prod_{j=1}^m \frac{\partial \tau (z_j, c(\bz_{1:j-1})) }{\partial z_j}
		\]
		\vspace{-0.3cm}
	\end{block}
	
	\myfootnotewithlink{https://arxiv.org/abs/1912.02762}{Papamakarios G. et al. Normalizing flows for probabilistic modeling and inference, 2019} 
\end{frame}
%=======
\begin{frame}{Autoregressive flows}
	\begin{block}{Forward and inverse transforms}
		\[
			x_j = \tau (z_j, c(\bz_{1:j-1})) \quad \Leftrightarrow \quad z_j = \tau^{-1} (x_j, c(\bz_{1:j-1}))
		\]
		\begin{figure}
			\includegraphics[width=\linewidth]{figs/autoregressive_flow}
		\end{figure}
	\end{block}
	\begin{itemize}
		\item Forward transform is \textbf{not sequential}.
		\item Inverse transform is \textbf{sequential}.
	\end{itemize}
		
	\myfootnotewithlink{https://arxiv.org/abs/1912.02762}{Papamakarios G. et al. Normalizing flows for probabilistic modeling and inference, 2019} 
\end{frame}
%=======
\section{Inverse autoregressive flows}
%=======
\begin{frame}{Inverse autoregressive flow (IAF)}
	Let use the following reparametrization:
	$\tilde{\bsigma} = \frac{1}{\bsigma}$; $ \tilde{\bmu} = - \frac{\bmu}{\bsigma}$.
	
	\begin{block}{Gaussian autoregressive flow}
		\vspace{-0.5cm}
		\begin{align*}
			x_j &= \sigma_j (\bx_{1:j-1}) \cdot z_j + \mu_j(\bx_{1:j-1}) =  \left( z_j - \tilde{\mu}_j(\bx_{1:j-1})\right) \cdot \frac{1}{\tilde{\sigma}_j (\bx_{1:j-1}) }\\
			z_j &= \left(x_j - \mu_j(\bx_{1:j-1}) \right) \cdot \frac{1}{ \sigma_j (\bx_{1:j-1})} = \tilde{\sigma}_j (\bx_{1:j-1}) \cdot x_j + \tilde{\mu}_j(\bx_{1:j-1}).
		\end{align*}
		\vspace{-0.3cm}
	\end{block}
	Let just swap $\bz$ and $\bx$. 
	
	\begin{block}{Inverse autoregressive flow}
		\vspace{-0.5cm}
		\begin{align*}
			\bx &= g(\bz, \btheta) \quad \Rightarrow \quad x_j = \tilde{\sigma}_j (\bz_{1:j-1}) \cdot z_j + \tilde{\mu}_j(\bz_{1:j-1}) \\
			\bz &= f(\bx, \btheta) \quad \Rightarrow \quad z_j = \left( x_j - \tilde{\mu}_j(\bz_{1:j-1})\right) \cdot \frac{1}{\tilde{\sigma}_j (\bz_{1:j-1}) }.
		\end{align*}
		\vspace{-0.3cm}
	\end{block}
	
	\myfootnotewithlink{https://arxiv.org/abs/1606.04934}{Kingma D. P. et al. Improving Variational Inference with Inverse Autoregressive Flow, 2016} 
\end{frame}
%=======
\begin{frame}{Inverse autoregressive flow (IAF)}
	
	\begin{minipage}[t]{0.65\columnwidth}
		\begin{block}{Gaussian autoregressive flow: $f(\bx, \btheta)$}
			\[
				x_j = \sigma_j (\bx_{1:j-1}) \cdot z_j + \mu_j(\bx_{1:j-1}).
			\]
		\end{block}
	\end{minipage}%
	\begin{minipage}[t]{0.35\columnwidth}
		\begin{figure}[h]
			\centering
			\includegraphics[width=.9\linewidth]{figs/maf_iaf_explained_1.png}
		\end{figure}
	\end{minipage} \\
	
	\begin{minipage}[t]{0.65\columnwidth}
		\begin{block}{Inverse transform: $g(\bz, \btheta)$}
			\vspace{-0.5cm}
			\begin{align*}
				z_j &= (x_j - \mu_j(\bx_{1:j-1})) \cdot \frac{1}{\sigma_j (\bx_{1:j-1}) }; \\
				z_j &= \tilde{\sigma}_j (\bx_{1:j-1}) \cdot x_j + \tilde{\mu}_j(\bx_{1:j-1}).
			\end{align*}
			\vspace{-0.4cm}
		\end{block}
	\end{minipage}% 
	\begin{minipage}[t]{0.35\columnwidth}
		\begin{figure}[h]
			\centering
			\includegraphics[width=.9\linewidth]{figs/maf_iaf_explained_2.png}
		\end{figure}
	\end{minipage}\\
	\vspace{0.1cm}
	
	\begin{minipage}[t]{0.65\columnwidth}
		\begin{block}{Inverse autoregressive flow: $f(\bx, \btheta)$}
			\[
			x_j = \tilde{\sigma}_j (\bz_{1:j-1}) \cdot z_j + \tilde{\mu}_j(\bz_{1:j-1}).
			\]
		\end{block}
	\end{minipage}%
	\begin{minipage}[t]{0.35\columnwidth}
		\begin{figure}[h]
			\centering
			\includegraphics[width=.9\linewidth]{figs/maf_iaf_explained_3.png}
		\end{figure}
	\end{minipage}
	
	\myfootnotewithlink{https://blog.evjang.com/2018/01/nf2.html}{image credit: https://blog.evjang.com/2018/01/nf2.html}
\end{frame}
%=======
\begin{frame}{Autoregressive flows}
	\begin{block}{Forward and inverse transforms in MAF}
		\vspace{-0.6cm}
		\begin{align*}
			\bx &= g(\bz, \btheta) \quad \Rightarrow \quad x_j = \sigma_j (\bx_{1:j-1}) \cdot z_j + \mu_j(\bx_{1:j-1}). \\
			\bz &= f(\bx, \btheta) \quad \Rightarrow \quad z_j = \left(x_j - \mu_j(\bx_{1:j-1}) \right) \cdot \frac{1}{\sigma_j (\bx_{1:j-1}) }.
		\end{align*}
		\vspace{-0.6cm}
		\begin{itemize}
			\item Sampling is sequential.
			\item Density estimation is parallel.
		\end{itemize}
	\end{block}
	\begin{block}{Forward and inverse transforms in IAF}
		\vspace{-0.6cm}
		\begin{align*}
			\bx &= g(\bz, \btheta) \quad \Rightarrow \quad x_j = \tilde{\sigma}_j (\bz_{1:j-1}) \cdot z_j + \tilde{\mu}_j(\bz_{1:j-1}). \\
			\bz &= f(\bx, \btheta) \quad \Rightarrow \quad z_j = \left(x_j - \tilde{\mu}_j(\bz_{1:j-1}) \right) \cdot \frac{1}{\tilde{\sigma}_j (\bz_{1:j-1})}.
		\end{align*}
		\vspace{-0.6cm}
		\begin{itemize}
			\item Sampling is parallel.
			\item Density estimation is sequential.
		\end{itemize}
	\end{block}
	\myfootnotewithlink{https://arxiv.org/abs/1705.07057}{Papamakarios G., Pavlakou T., Murray I. Masked Autoregressive Flow for Density Estimation, 2017} 
\end{frame}
%=======
\begin{frame}{Autoregressive flows}
	\begin{figure}
		\includegraphics[width=0.8\linewidth]{figs/flows_how2.png}
	\end{figure}
	\begin{itemize}	
		\item MAF performs parallel inference that is useful for density estimation tasks (forward KL or MLE).
		\item IAF performs parallel generation that is useful for optimization of reverse KL.
	\end{itemize}
	
	\myfootnotewithlink{https://arxiv.org/abs/1605.08803}{Dinh L., Sohl-Dickstein J., Bengio S. Density estimation using Real NVP, 2016} 
\end{frame}
%=======
\section{RealNVP: coupling layer}
%=======
\begin{frame}{RealNVP}
	\begin{block}{Coupling layer}
		\vspace{-0.8cm}
		\begin{equation*}
			\begin{cases} \bz_{1:d} = \bx_{1:d}; \\ \bz_{d:m} = \tau (\bx_{d:m}, c(\bx_{1:d}));\end{cases} 
			\quad \Leftrightarrow \quad 
			\begin{cases} \bx_{1:d} = \bz_{1:d}; \\ \bx_{d:m} = \tau^{-1} (\bz_{d:m}, c(\bz_{1:d})).\end{cases}
		\end{equation*}
		\vspace{-0.5cm}
	\end{block}
	\begin{block}{Image partitioning}
		\begin{figure}
			\centering
			\includegraphics[width=0.65\linewidth]{figs/realnvp_masking.png}
		\end{figure}
	\end{block}
	\vspace{-0.5cm}
	
	Checkerboard ordering uses masking, channelwise ordering uses splitting.
	\myfootnotewithlink{https://arxiv.org/abs/1410.8516}{Dinh L., Krueger D., Bengio Y. NICE: Non-linear Independent Components Estimation, 2014}
\end{frame}
%=======
\begin{frame}{RealNVP}
	\begin{block}{Affine coupling law}
		\[
			\begin{cases} \bz_{1:d} = \bx_{1:d}; \\ \bz_{d:m} = \bx_{d:m} \odot c_1(\bx_{1:d}, \btheta) + c_2(\bx_{1:d}, \btheta).\end{cases} 
		\]
		\[
			\begin{cases} \bx_{1:d} = \bz_{1:d}; \\ \bx_{d:m} = \left(\bz_{d:m} - c_2(\bz_{1:d}, \btheta) \right) \cdot \frac{1}{c_1(\bz_{1:d}, \btheta)}.\end{cases}
		\]
	\end{block}
	\begin{block}{Jacobian}
		\vspace{-0.5cm}
		\[
		\det \left( \frac{\partial \bz}{\partial \bx} \right) = \det 
		\begin{pmatrix}
			\bI_d & 0_{d \times m - d} \\
			\frac{\partial \bz_{d:m}}{\partial \bx_{1:d}} & \frac{\partial \bz_{d:m}}{\partial \bx_{d:m}}
		\end{pmatrix} = \prod_{j=1}^{m - d} c_1(\bx_{1:d}, \btheta)_j.
		\]
		Non-Volume Preserving (the determinant of Jacobian $\neq 1$).
	\end{block}
	
	\myfootnotewithlink{https://arxiv.org/abs/1605.08803}{Dinh L., Sohl-Dickstein J., Bengio S. Density estimation using Real NVP, 2016} 
\end{frame}
%=======
\begin{frame}{MAF vs IAF vs RealNVP}
	\begin{block}{MADE/MAF}
		\vspace{-0.5cm}
		\[
		\bx = \bsigma (\bx) \odot \bz + \bmu(\bx).
		\]
		Estimating the density $p(\bx | \btheta)$ - 1 pass, sampling - $m$ passes.
	\end{block}
	\begin{block}{IAF}
		\vspace{-0.5cm}
		\[
		\bx = \tilde{\bsigma} (\bz) \odot \bz + \tilde{\bmu}(\bz).
		\]
		Estimating the density $p(\bx | \btheta)$ - $m$ passes, sampling - 1 pass.
	\end{block}
	\begin{block}{RealNVP}
		\vspace{-0.2cm}
		\[
			\begin{cases}
				\bx_{1:d} &= \bz_{1:d}; \\ 
				\bx_{d:m} &= \bz_{d:m} \odot c_1(\bz_{1:d}, \btheta) + c_2(\bz_{1:d}, \btheta).
			\end{cases}
		\]
		\vspace{-0.5cm}
		Estimating the density $p(\bx | \btheta)$ - 1 pass, sampling - 1 pass.
	\end{block}
	\myfootnotewithlink{https://arxiv.org/abs/1705.07057}{Papamakarios G., Pavlakou T., Murray I. Masked Autoregressive Flow for Density Estimation, 2017} 
\end{frame}
%=======
\begin{frame}{MAF vs IAF vs RealNVP}
	\begin{block}{RealNVP}
		\vspace{-0.3cm}
		\[
			\begin{cases}
				\bx_{1:d} &= \bz_{1:d}; \\ 
				\bx_{d:m} &= \bz_{d:m} \odot c_1(\bz_{1:d}, \btheta) + c_2(\bz_{1:d}, \btheta).
			\end{cases}
		\]
		\vspace{-0.3cm}
	\end{block}
	\begin{itemize}
		\item Calculating the density $p(\bx | \btheta)$ - 1 pass.
		\item Sampling - 1 pass.
	\end{itemize}
	
	RealNVP is a special case of MAF and IAF:
	\begin{block}{MAF}
		\vspace{-0.5cm}
		\begin{equation*}
			\begin{cases}
				\mu_j  = 0, \sigma_j = 1, \, j = 1, \dots, d; \\
				\mu_j, \sigma_j \text{ -- functions of } \bx_{1:d}, \, j = d+1, \dots, m.
			\end{cases}
		\end{equation*}
		\vspace{-0.3cm}
	\end{block}
	\begin{block}{IAF}
		\vspace{-0.3cm}
		\begin{equation*}
			\begin{cases}
				\tilde{\mu}_j = 0, \tilde{\sigma}_j = 1, \, j = 1, \dots, d; \\
				\tilde{\mu}_j, \tilde{\sigma}_j \text{ -- functions of } \bz_{1:d}, \, j = d+1, \dots, m.
			\end{cases}
		\end{equation*}
	\end{block}
	\myfootnotewithlink{https://arxiv.org/abs/1705.07057}{Papamakarios G., Pavlakou T., Murray I. Masked Autoregressive Flow for Density Estimation, 2017} 
\end{frame}
%=======
\begin{frame}{RealNVP samples}
		\begin{figure}
			\centering
			\includegraphics[width=\linewidth]{figs/realnvp_output.png}
		\end{figure}
	\myfootnotewithlink{https://arxiv.org/abs/1605.08803}{Dinh L., Sohl-Dickstein J., Bengio S. Density estimation using Real NVP, 2016} 
\end{frame}
%=======
\begin{frame}{Linear flows}
	\begin{block}{RealNVP}
		\vspace{-0.5cm}
		\begin{equation*}
			\begin{cases} \bz_{1:d} = \bx_{1:d}; \\ \bz_{d:m} = \tau (\bx_{d:m}, c(\bx_{1:d}));\end{cases} 
			\quad \Leftrightarrow \quad 
			\begin{cases} \bx_{1:d} = \bz_{1:d}; \\ \bx_{d:m} = \tau^{-1} (\bz_{d:m}, c(\bz_{1:d})).\end{cases}
		\end{equation*}
		\vspace{-0.2cm}
	\end{block}
	\begin{itemize}
	\item First step is a \textbf{split} operator which decouples a variable into 2 subparts: $\bx_1$ and $\bx_2$ (usualy channel-wise).
	\item We should \textbf{permute} components between different layers.
	\end{itemize}
	\[
		\bz = \bW \bx, \quad \bW \in \bbR^{m \times m}
	\]
	In general, we need $O(m^3)$ to invert matrix.
	\begin{block}{Invertibility}
		\begin{itemize}
			\item Diagonal matrix $O(m)$.
			\item Triangular matrix $O(m^2)$.
			\item It is impossible to parametrize all invertible matrices.
		\end{itemize}
	\end{block}
\end{frame}
%=======
\begin{frame}{Glow samples}
	\begin{figure}
		\centering
		\includegraphics[width=\linewidth]{figs/glow_faces.png}
	\end{figure}
	\myfootnotewithlink{https://arxiv.org/abs/1807.03039}{Kingma D. P., Dhariwal P. Glow: Generative Flow with Invertible 1x1 Convolutions, 2018}
\end{frame}
%=======
\begin{frame}{Summary}
	\begin{itemize}
		\item Gaussian autoregressive model is an autoregressive flow with triangular Jacobian.
		\vfill
		\item Inverse autoregressive flow is able to sample fast, but the inference is slow.
		\vfill
		\item MAF/IAF is a special case of autoregressive flows.
		\vfill
		\item The RealNVP is an effective type of flow (special case of AR flows) that uses coupling layer.
	\end{itemize}
\end{frame}
%=======
\end{document} 