\input{../utils/preamble}
\createdgmtitle{13}

\usepackage{tikz}

\usetikzlibrary{arrows,shapes,positioning,shadows,trees}
%--------------------------------------------------------------------------------
\begin{document}
%--------------------------------------------------------------------------------
\begin{frame}[noframenumbering,plain]
%\thispagestyle{empty}
\titlepage
\end{frame}
%=======
\begin{frame}{Recap of previous lecture}
	\begin{figure}
		\centering
		\includegraphics[width=0.85\linewidth]{figs/vqvae}
	\end{figure}
	\vspace{-0.3cm}
	\begin{block}{Deterministic variational posterior}
		\vspace{-0.8cm}
		\[
			q(c_{ij} = k^* | \bx, \bphi) = \begin{cases}
				1 , \quad \text{for } k^* = \argmin_k \| [\bz_e]_{ij} - \be_k \|; \\
				0, \quad \text{otherwise}.
			\end{cases}
		\]
		\vspace{-0.8cm}
	\end{block}
	\begin{block}{ELBO}
		\vspace{-0.6cm}
		\[
			\mathcal{L} (\bphi, \btheta)  = \mathbb{E}_{q(c | \bx, \bphi)} \log p(\bx | \be_{c} , \btheta) - \log K =  \log p(\bx | \bz_q, \btheta) - \log K.
		\]
		\vspace{-0.6cm}
	\end{block}
	\begin{block}{Straight-through gradient estimation}
		\vspace{-0.6cm}
		\[
			\frac{\partial \log p(\bx | \bz_q , \btheta)}{\partial \bphi} = \frac{\partial \log p(\bx | \bz_q, \btheta)}{\partial \bz_q} \cdot {\color{red}\frac{\partial \bz_q}{\partial \bphi}} \approx \frac{\partial \log p(\bx | \bz_q, \btheta)}{\partial \bz_q} \cdot \frac{\partial \bz_e}{\partial \bphi}
		\]
	\end{block}
	\myfootnotewithlink{https://arxiv.org/abs/1711.00937}{Oord A., Vinyals O., Kavukcuoglu K. Neural Discrete Representation Learning, 2017} 
\end{frame}
%=======
\begin{frame}{Recap of previous lecture}
	\vspace{-0.4cm}
	\begin{block}{Gumbel-max trick}
		Let $g_k \sim \text{Gumbel}(0, 1)$ for $k = 1, \dots, K$. Then
		\vspace{-0.3cm}
		\[
			c = \argmax_k [\log \pi_k + g_k]
		\]
		\vspace{-0.6cm} \\
		has a categorical distribution $c \sim \text{Categorical}(\bpi)$.
	\end{block}
	\vspace{-0.2cm}
	\begin{block}{Gumbel-softmax relaxation}
		{\color{violet}Con}{\color{teal}crete} distribution = {\color{violet}\textbf{con}tinuous} + {\color{teal}dis\textbf{crete}}
		\vspace{-0.2cm}
		\[
			\hat{c}_k = \frac{\exp \left(\frac{\log q(k | \bx, \bphi) + g_k}{\tau}\right)}{\sum_{j=1}^K \exp \left(\frac{\log q(j | \bx, \bphi) + g_j}{\tau}\right)}, \quad k = 1, \dots, K.
		\]
		\vspace{-0.7cm}
 	\end{block}
	\begin{block}{Reparametrization trick}
		\vspace{-0.4cm}
		\[
			\nabla_{\bphi} \mathbb{E}_{q(c | \bx, \bphi)} \log p(\bx | \be_{c} , \btheta) = \bbE_{\text{Gumbel}(0, 1)} \nabla_{\bphi} \log p(\bx | \bz , \btheta),
		\]
		where $\bz = \sum_{k=1}^K\hat{c}_k \be_k$ (all operations are differentiable now).
	\end{block}
 	\vspace{-0.2cm}
	\myfootnote{
	\href{https://arxiv.org/abs/1611.00712}{Maddison C. J., Mnih A., Teh Y. W. The Concrete distribution: A continuous relaxation of discrete random variables, 2016} \\
	\href{https://arxiv.org/abs/1611.01144}{Jang E., Gu S., Poole B. Categorical reparameterization with Gumbel-Softmax, 2016}
	}
\end{frame}
%=======
\begin{frame}{Recap of previous lecture}
	Consider Ordinary Differential Equation    
	\begin{align*}
	    \frac{d \bz(t)}{dt} &= f(\bz(t), t, \btheta); \quad \text{with initial condition }\bz(t_0) = \bz_0. \\
	    \bz(t_1) &= \int^{t_1}_{t_0} f(\bz(t), t, \btheta) d t  + \bz_0 = \text{ODESolve}(\bz(t_0), f, t_0,t_1, \btheta).
	\end{align*}
	\vspace{-0.4cm}
	\begin{block}{Euler update step}
		\vspace{-0.6cm}
		\[
			\frac{\bz(t + \Delta t) - \bz(t)}{\Delta t} = f(\bz(t), t, \btheta) \,\, \Rightarrow \,\, \bz(t + \Delta t) = \bz(t) + \Delta t \cdot f(\bz(t), t, \btheta)
		\]
		\vspace{-0.7cm}
	\end{block}
	\begin{block}{Residual block}
		\vspace{-0.4cm}
		\[
			\bz_{t + 1} = \bz_t + f(\bz_t, \btheta)
		\]
		It is equivalent to Euler update step for solving ODE with $\Delta t = 1$!
	\end{block}
	In the limit of adding more layers and taking smaller steps we get: 
	\[
	    \frac{d \bz(t)}{dt} = f(\bz(t), t, \btheta); \quad \bz(t_0) = \bx; \quad \bz(t_1) = \by.
	\]

	\myfootnotewithlink{https://arxiv.org/abs/1806.07366}{Chen R. T. Q. et al. Neural Ordinary Differential Equations, 2018}   
\end{frame}
%=======
\begin{frame}{Recap of previous lecture}
	\begin{block}{Forward pass (loss function)}
		\vspace{-0.8cm}
		\begin{align*}
			L(\by) = L(\bz(t_1)) &= L\left( \bz(t_0) + \int_{t_0}^{t_1} f(\bz(t), t, \btheta) dt \right) \\ &= L\bigl(\text{ODESolve}(\bz(t_0), f, t_0,t_1, \btheta) \bigr)
		\end{align*}
	\vspace{-0.5cm}
	\end{block}
	\textbf{Note:} ODESolve could be any method (Euler step, Runge-Kutta methods).
	\begin{block}{Backward pass (gradients computation)}
		For fitting parameters we need gradients:
		\[
			\ba_{\bz}(t) = \frac{\partial L(\by)}{\partial \bz(t)}; \quad \ba_{\btheta}(t) = \frac{\partial L(\by)}{\partial \btheta(t)}.
		\]
		In theory of optimal control these functions called \textbf{adjoint} functions. 
		They show how the gradient of the loss depends on the hidden state~$\bz(t)$ and parameters $\btheta$.
	\end{block}

	\myfootnotewithlink{https://arxiv.org/abs/1806.07366}{Chen R. T. Q. et al. Neural Ordinary Differential Equations, 2018}     
\end{frame}
%=======
\begin{frame}{Outline}
	\tableofcontents
\end{frame}
%=======
\section{Neural ODE}
%=======
\begin{frame}{Neural ODE}
	\vspace{-0.3cm}
	\begin{block}{Adjoint functions}
		\vspace{-0.3cm}
		\[
		\ba_{\bz}(t) = \frac{\partial L(\by)}{\partial \bz(t)}; \quad \ba_{\btheta}(t) = \frac{\partial L(\by)}{\partial \btheta(t)}.
		\]
		\vspace{-0.6cm}
	\end{block}
	\begin{block}{Theorem (Pontryagin)}
		\vspace{-0.6cm}
		\[
		\frac{d \ba_{\bz}(t)}{dt} = - \ba_{\bz}(t)^T \cdot \frac{\partial f(\bz(t), t, \btheta)}{\partial \bz}; \quad \frac{d \ba_{\btheta}(t)}{dt} = - \ba_{\bz}(t)^T \cdot \frac{\partial f(\bz(t),  t, \btheta)}{\partial \btheta}.
		\]
		Do we know any initilal condition?
	\end{block}
	\begin{block}{Solution for adjoint function}
		\vspace{-0.8cm}
		\begin{align*}
			\frac{\partial L}{\partial \btheta(t_0)} &= \ba_{\btheta}(t_0) =  - \int_{t_1}^{t_0} \ba_{\bz}(t)^T \frac{\partial f(\bz(t), t, \btheta)}{\partial \btheta(t)} dt + 0\\
			\frac{\partial L}{\partial \bz(t_0)} &= \ba_{\bz}(t_0) =  - \int_{t_1}^{t_0} \ba_{\bz}(t)^T \frac{\partial f(\bz(t), t, \btheta)}{\partial \bz(t)} dt + \frac{\partial L}{\partial \bz(t_1)}\\
		\end{align*}
		\vspace{-1.2cm}
	\end{block}
	\textbf{Note:} These equations are solved back in time.
	\myfootnotewithlink{https://arxiv.org/abs/1806.07366}{Chen R. T. Q. et al. Neural Ordinary Differential Equations, 2018}   
\end{frame}
%=======
\begin{frame}{Neural ODE}
	\vspace{-0.2cm}
	\begin{block}{Forward pass}
		\vspace{-0.5cm}
		\[
		\bz(t_1) = \int^{t_1}_{t_0} f(\bz(t), t, \btheta) d t  + \bz_0 \quad \Rightarrow \quad \text{ODE Solver}
		\]
		\vspace{-0.6cm}
	\end{block}
	\begin{block}{Backward pass}
		\vspace{-0.8cm}
		\begin{equation*}
			\left.
			{\footnotesize 
				\begin{aligned}
					\frac{\partial L}{\partial \btheta(t_0)} &= \ba_{\btheta}(t_0) =  - \int_{t_1}^{t_0} \ba_{\bz}(t)^T \frac{\partial f(\bz(t), t, \btheta)}{\partial \btheta(t)} dt + 0 \\
					\frac{\partial L}{\partial \bz(t_0)} &= \ba_{\bz}(t_0) =  - \int_{t_1}^{t_0} \ba_{\bz}(t)^T \frac{\partial f(\bz(t), t, \btheta)}{\partial \bz(t)} dt + \frac{\partial L}{\partial \bz(t_1)} \\
					\bz(t_0) &= - \int^{t_0}_{t_1} f(\bz(t), t, \btheta) d t  + \bz_1.
				\end{aligned}
			}
			\right\rbrace
			\Rightarrow
			\text{ODE Solver}
		\end{equation*}
		\vspace{-0.4cm} 
	\end{block}
	\textbf{Note:} These scary formulas are the standard backprop in the discrete case.
	\begin{figure}
		\centering
		\includegraphics[width=\linewidth]{figs/neural_ode}
	\end{figure}
	\myfootnotewithlink{https://arxiv.org/abs/1806.07366}{Chen R. T. Q. et al. Neural Ordinary Differential Equations, 2018}   
\end{frame}
%=======
\section{Continuous-in-time normalizing flows}
%=======
\begin{frame}{Continuous-in-time Normalizing Flows}
	\vspace{-0.3cm}
	\begin{block}{Discrete-in-time NF}
		\vspace{-0.8cm}
		  \[
		  \bz_{t+1} = f(\bz_t, \btheta); \quad \log p(\bz_{t+1}) = \log p(\bz_{t}) - \log \left| \det \frac{\partial f(\bz_t, \btheta)}{\partial \bz_{t}} \right| .
		  \]
		\vspace{-0.7cm}
	\end{block}
	\begin{block}{Continuous-in-time dynamics}
		\vspace{-0.2cm}
		\[
			\frac{d\bz(t)}{dt} = f(\bz(t), t, \btheta).
		\]
	\end{block}
	\vspace{-0.6cm}
	\begin{minipage}[t]{0.4\columnwidth}
		\begin{figure}
			\centering
			\includegraphics[width=0.75\linewidth]{figs/cnf_flow.png}
		\end{figure}
	\end{minipage}%
	\begin{minipage}[t]{0.6\columnwidth}
		\begin{figure}
			\centering
			\includegraphics[width=0.8\linewidth]{figs/ffjord.png}
		\end{figure}
	\end{minipage}
	\myfootnotewithlink{https://arxiv.org/abs/1810.01367}{Grathwohl W. et al. FFJORD: Free-form Continuous Dynamics for Scalable Reversible Generative Models, 2018}  
\end{frame}
%=======
\begin{frame}{Continuous-in-time Normalizing Flows}
	\begin{block}{Theorem (Picard)}
		If $f$ is uniformly Lipschitz continuous in $\bz$ and continuous in $t$, then the ODE has a \textbf{unique} solution.
	\end{block}
	\textbf{Note:} Unlike discrete-in-time flows, $f$ does not need to be bijective (uniqueness guarantees bijectivity).
	\begin{itemize}
		\item Discrete-in-time normalizing flows need invertible $f$. Here we have sequence of $\log p(\bz_t)$.
		\item Continuous-in-time flows require only smoothness of $f$. Here we need to get $\log(p(\bz(t), t))$
	\end{itemize}
	\begin{block}{Forward and inverse transforms}
		\vspace{-0.7cm}
		\begin{align*}
			\bx &= \bz(t_1) = \bz(t_0) + \int_{t_0}^{t_1} f(\bz(t), t, \btheta) dt \\
			\bz &= \bz(t_0) = \bz(t_1) + \int_{t_1}^{t_0} f(\bz(t), t, \btheta) dt
		\end{align*}
		\vspace{-0.7cm}
	\end{block}
	\myfootnotewithlink{https://arxiv.org/abs/1806.07366}{Chen R. T. Q. et al. Neural Ordinary Differential Equations, 2018}   
\end{frame}
%=======
\begin{frame}{Continuous-in-time Normalizing Flows}
	\vspace{-0.3cm}
	\begin{block}{Theorem (Kolmogorov-Fokker-Planck: special case)}
		If $f$ is uniformly Lipschitz continuous in $\bz$ and continuous in $t$, then
		\[
			\frac{d \log p(\bz(t), t)}{d t} = - \text{tr} \left( \frac{\partial f (\bz(t), t, \btheta)}{\partial \bz(t)} \right).
		\]
	\end{block}
	\vspace{-0.6cm}
	\[
		\log p(\bx | \btheta) = \log p(\bz) - \int_{t_0}^{t_1} \text{tr}  \left( \frac{\partial f (\bz(t), t, \btheta)}{\partial \bz(t)} \right) dt.
	\]
	Here $p(\bx | \btheta) = p(\bz(t_1), t_1)$, $p(\bz) = p(\bz(t_0), t_0)$.
	\textbf{Adjoint} method is used for getting the derivatives.
	\begin{block}{Forward transform + log-density}
		\vspace{-0.3cm}
		\[
			\begin{bmatrix}
				\bx \\
				\log p(\bx | \btheta)
			\end{bmatrix}
			= 
			\begin{bmatrix}
				\bz \\
				\log p(\bz)
			\end{bmatrix} + 
			\int_{t_0}^{t_1} 
			\begin{bmatrix}
				f(\bz(t), t, \btheta) \\
				- \text{tr} \left( \frac{\partial f(\bz(t), t, \btheta)}{\partial \bz(t)} \right) 
			\end{bmatrix} dt.
		\]
		\vspace{-0.4cm}
	\end{block}
	It costs $O(m^2)$ to get the trace of the Jacobian (evaluation of determinant of the Jacobian costs $O(m^3)$!).
	\myfootnotewithlink{https://arxiv.org/abs/1806.07366}{Chen R. T. Q. et al. Neural Ordinary Differential Equations, 2018}   
\end{frame}
%=======
\begin{frame}{Continuous-in-time Normalizing Flows}
	\vspace{0.2cm}
	\begin{itemize}
		\item $\text{tr} \left( \frac{\partial f(\bz(t), \btheta)}{\partial \bz(t)} \right)$ costs $O(m^2)$ ($m$
		evaluations of $f$), since we have to compute a derivative for each diagonal element. 
		\item Jacobian vector products ${\color{violet}\bv^T \frac{\partial f}{\partial \bz}}$ can be computed for approximately the same cost as evaluating $f$.
	\end{itemize}
	It is possible to reduce cost from $O(m^2)$ to $O(m)$!
	\begin{block}{Hutchinson's trace estimator}
		If $\bepsilon \in \bbR^m$ is a random variable with $\mathbb{E} [\bepsilon] = 0$ and $\text{Cov} (\bepsilon) = I$, then
		\vspace{-0.3cm}
		\[
		    \text{tr}(\mathbf{A}) = \text{tr}\left(\mathbf{A}\mathbb{E}_{p(\bepsilon)} \left[ \bepsilon \bepsilon^T \right]\right) =  \mathbb{E}_{p(\bepsilon)} \left[  \text{tr}\left(  \mathbf{A}  \bepsilon \bepsilon^T \right) \right] =  \mathbb{E}_{p(\bepsilon)} \left[ {\color{violet} \bepsilon^T \mathbf{A}} \bepsilon  \right]
		\]
		\vspace{-0.6 cm}
	\end{block}
	\begin{block}{FFJORD density estimation}
		\vspace{-0.8cm}
		\begin{multline*}
		    \log p(\bz(t_1)) = \log p(\bz(t_0)) - \int_{t_0}^{t_1} \text{tr}  \left( \frac{\partial f (\bz(t), t, \btheta)}{\partial \bz(t)} \right) dt = \\ = \log p(\bz(t_0)) - \mathbb{E}_{p(\bepsilon)} \int_{t_0}^{t_1} \left[ {\color{violet}\bepsilon^T \frac{\partial f}{\partial \bz}} \bepsilon \right] dt.
		\end{multline*}
	\end{block}
	\myfootnotewithlink{https://arxiv.org/abs/1810.01367}{Grathwohl W. et al. FFJORD: Free-form Continuous Dynamics for Scalable Reversible Generative Models, 2018} 
\end{frame}
%=======
\section{Langevin dynamic and SDE basics}
%=======
\begin{frame}{Langevin dynamic}
	Imagine that we have some generative model $p(\bx | \btheta)$.
	\begin{block}{Statement}
		Let $\bx_0$ be a random vector. Then under mild regularity conditions for small enough $\eta$ samples from the following dynamics
		\[
			\bx_{t + 1} = \bx_t + \eta \frac{1}{2} \nabla_{\bx_t} \log p(\bx_t | \btheta) + \sqrt{\eta} \cdot \bepsilon, \quad \bepsilon \sim \cN(0, 1).
		\]
		will comes from $p(\bx | \btheta)$.
	\end{block}
	What do we get if $\bepsilon = \boldsymbol{0}$?
	\begin{block}{Energy-based model}
		\vspace{-0.4cm}
		\[
			p(\bx | \btheta) = \frac{\hat{p}(\bx | \btheta)}{Z_{\btheta}}, \quad \text{where } Z_{\btheta} = \int \hat{p}(\bx | \btheta) d \bx
		\]
		\[
			\nabla_{\bx} \log p(\bx | \btheta) = \nabla_{\bx} \log \hat{p}(\bx | \btheta) - \nabla_{\bx} \log Z_{\btheta} = \nabla_{\bx} \log \hat{p}(\bx | \btheta)
		\]
		Gradient of normalized density equals to gradient of unnormalized density.
	\end{block}
	\myfootnotewithlink{https://www.stats.ox.ac.uk/~teh/research/compstats/WelTeh2011a.pdf}{Welling M. Bayesian Learning via Stochastic Gradient Langevin Dynamics, 2011}
\end{frame}
%=======
\begin{frame}{Stochastic differential equation (SDE)}
	Let define stochastic process $\bx(t)$ with initial condition $\bx(0) \sim p_0(\bx)$:
	\[
		d\bx = \mathbf{f}(\bx, t) dt + g(t) d \bw
	\]
	\vspace{-0.6cm}
	\begin{itemize}
		 \item $\mathbf{f}(\bx, t)$ is the \textbf{drift} function of $\bx(t)$.
		 \item $g(t)$ is the \textbf{diffusion} coefficient of $\bx(t)$.
		 \item If $g(t) = 0$ we get standard ODE.
		 \item $\bw(t)$ is the standard Wiener process (Brownian motion)
		 \[		
		 \bw(t) - \bw(s) \sim \cN(0, t - s), \quad d \bw = \bepsilon \cdot \sqrt{dt}, \, \text{where } \bepsilon \sim \cN(0, 1).
		 \]
	\end{itemize}
	 How to get distribution $p(\bx, t)$ for $\bx(t)$?
 	\begin{block}{Theorem (Kolmogorov-Fokker-Planck)}
 		Evolution of the distribution $p(\bx, t)$ is given by the following ODE:
 		\vspace{-0.2cm}
 		\[
 			\frac{\partial p(\bx, t)}{\partial t} = \text{tr}\left(- \frac{\partial}{\partial \bx} \bigl[ \mathbf{f}(\bx, t) p(\bx, t)\bigr] + \frac{1}{2} g^2(t) \frac{\partial^2 p(\bx, t)}{\partial \bx^2} \right)
 		\]
 		\vspace{-0.6cm}
 	\end{block}
\end{frame}
%=======
\begin{frame}{Stochastic differential equation (SDE)}
	\[
		d\bx = {\color{violet}\mathbf{f}(\bx, t)} dt + {\color{teal}g(t)} d \bw, \quad d \bw = \bepsilon \cdot \sqrt{dt}, \quad \bepsilon \sim \cN(0, 1).
	\]
	\vspace{-0.4cm}
	\begin{block}{Langevin SDE (special case)}
		\vspace{-0.3cm}
		\[
			d \bx = {\color{violet}\frac{1}{2} \frac{\partial}{\partial \bx} \log p(\bx, t)} d t + {\color{teal} 1 }d \bw
		\]
		\vspace{-0.6cm}
	\end{block}
	\begin{block}{Langevin discrete dynamic}
		\vspace{-0.3cm}
		\[
			\bx_{t + 1} = \bx_t + \eta \frac{1}{2} \frac{\partial}{\partial \bx} \log p(\bx | \btheta) + \sqrt{\eta} \cdot \bepsilon, \quad \eta \approx dt.
		\]
		\vspace{-0.3cm}
	\end{block}
	Let apply KFP theorem.
	\begin{multline*}
		\frac{\partial p(\bx, t)}{\partial t} =  \text{tr} \left(- \frac{\partial}{\partial \bx}\left[ {\color{olive}p(\bx, t) \frac{1}{2} \frac{\partial}{\partial \bx} \log p(\bx, t)} \right]  + \frac{1}{2} \frac{\partial^2 p(\bx, t)}{\partial \bx^2} \right) = \\
		= \text{tr} \left(- \frac{\partial}{\partial \bx}\left[ {\color{olive}\frac{1}{2} \frac{\partial}{\partial \bx} p(\bx, t) } \right]  + \frac{1}{2} \frac{\partial^2 p(\bx, t)}{\partial \bx^2} \right) = 0
	\end{multline*}
	The density $p(\bx, t) = \text{const}$.
\end{frame}
%=======
\begin{frame}{Summary}
	\begin{itemize}
		\item Adjoint method generalizes backpropagation procedure and allows to train Neural ODE solving ODE for adjoint function back in time.
		\vfill
		\item Kolmogorov-Fokker-Planck theorem allows to construct continuous-in-time normalizing flow with less functional restrictions.
		\vfill
		\item FFJORD model makes such kind of flows scalable.
		\vfill
		\item Langevin dynamics allows to sample from the model using the score function (due to the existence of stationary distribution for SDE).
	\end{itemize}
\end{frame}
\end{document} 