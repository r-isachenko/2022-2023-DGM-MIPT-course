\input{../utils/preamble}
\createdgmtitle{10}
%--------------------------------------------------------------------------------
\begin{document}
%--------------------------------------------------------------------------------
\begin{frame}[noframenumbering,plain]
%\thispagestyle{empty}
\titlepage
\end{frame}
%=======
\begin{frame}{Recap of previous lecture}
\vspace{-0.3cm}
	\begin{block}{ELBO objective}
		\vspace{-0.5cm}
		\[
			 \mathcal{L} (\bphi, \btheta)  = \mathbb{E}_{q(\bz | \bx, \bphi)} \left[\log p(\bx | \bz, \btheta) - KL(\log q(\bz| \bx, \bphi) || p(\bz)) \right] \rightarrow \max_{\bphi, \btheta}.
		\]	
		\vspace{-0.5cm}
	\end{block}
	What is the problem to make the variational posterior model an \textbf{implicit} model? \\
	 We have to estimate density ratio (in KL term)
	\[
		r(\bx, \bz) = \frac{q_1(\bx, \bz)}{q_2(\bx, \bz)} = \frac{q(\bz| \bx, \bphi) \pi (\bx)}{p(\bz) \pi(\bx)} = \frac{D(\bx, \bz)}{1 - D(\bx, \bz)}.
	\] 
	\begin{block}{Adversarial Variational Bayes}
		\vspace{-0.6cm}
		\[
			\max_D \left[ \bbE_{\pi(\bx)} \bbE_{q(\bz | \bx, \bphi)} \log D(\bx, \bz) + \bbE_{\pi(\bx)} \bbE_{p(\bz)} \log (1 - D(\bx, \bz)) \right]
		\]
	\end{block}
	\myfootnotewithlink{https://arxiv.org/abs/1701.04722}{Mescheder L., Nowozin S., Geiger A. Adversarial variational bayes: Unifying variational autoencoders and generative adversarial networks, 2017}
\end{frame}
%=======
\begin{frame}{Recap of previous lecture}
	\begin{block}{Main problems of standard GAN}
		\begin{itemize}
			\item Vanishing gradients (solution: non-saturating GAN);
			\item Mode collapse (caused by Jensen-Shannon divergence).
		\end{itemize}
	\end{block}
	\begin{block}{Standard GAN}
		\vspace{-0.7cm}
		\[
			\min_{\btheta} \max_{\bphi} \left[ \bbE_{\pi(\bx)} \log D(\bx, \bphi) + \bbE_{p(\bz)} \log (1 - D(G(\bz, \btheta), \bphi)) \right]
		\]
		\vspace{-0.7cm}
	\end{block}
	\vspace{-0.1cm}
	\begin{block}{Informal theoretical results}
		The real images distribution $\pi(\bx)$ and the generated images distribution $p(\bx | \btheta)$ are low-dimensional and have disjoint supports. In this case
		\vspace{-0.3cm}
		\[
			KL(\pi || p) = KL(p || \pi) = \infty, \quad JSD(\pi || p) = \log 2.
		\]
		\end{block}
		\myfootnote{\href{https://arxiv.org/abs/1406.2661}{Goodfellow I. J. et al. Generative Adversarial Networks, 2014} \\
		\href{https://arxiv.org/abs/1701.04862}{Arjovsky M., Bottou L. Towards Principled Methods for Training Generative Adversarial Networks, 2017}}
\end{frame}
%=======
\begin{frame}{Recap of previous lecture}
		\begin{block}{Wasserstein distance}
			\vspace{-0.5cm}
			\[
				W(\pi, p) = \inf_{\gamma \in \Gamma(\pi, p)} \bbE_{(\bx, \by) \sim \gamma} \| \bx - \by \| =  \inf_{\gamma \in \Gamma(\pi, p)} \int \| \bx - \by \| \gamma (\bx, \by) d \bx d \by
			\]
			\vspace{-0.5cm}
			\begin{itemize}
				\item $\gamma(\bx, \by)$ -- transportation plan (the amount of "dirt" that should be transported from point $\bx$ to point $\by$).
				\item $\Gamma(\pi, p)$ -- the set of all joint distributions $\Gamma (\bx, \by)$ with marginals $\pi$ and $p$ ($\int \gamma(\bx, \by) d \bx = p(\by)$, $\int \gamma(\bx, \by) d \by = \pi(\bx)$).
				\item $\gamma(\bx, \by)$ -- the amount, $\|\bx - \by \|$-- the distance.
			\end{itemize}
		\end{block}
		\begin{block}{Theorem (Kantorovich-Rubinstein duality)}
			\vspace{-0.2cm}
			\[
				W(\pi || p) = \frac{1}{K} \max_{\| f \|_L \leq K} \left[ \bbE_{\pi(\bx)} f(\bx)  - \bbE_{p(\bx)} f(\bx)\right],
			\]
			where $\| f \|_L \leq K$ are $K-$Lipschitz continuous functions ($f: \cX \rightarrow \bbR$).
		\end{block}
		\myfootnotewithlink{https://arxiv.org/abs/1701.07875}{Arjovsky M., Chintala S., Bottou L. Wasserstein GAN, 2017}
\end{frame}
%=======
\begin{frame}{Recap of previous lecture}
	\begin{block}{WGAN objective}
		\vspace{-0.5cm}
		\[
		\min_{\btheta} W(\pi || p) = \min_{\btheta} \max_{\bphi \in \boldsymbol{\Phi}} \left[ \bbE_{\pi(\bx)} f(\bx, \bphi)  - \bbE_{p(\bz)} f(G(\bz, \btheta), \bphi )\right].
		\]
		\vspace{-0.5cm}
	\end{block}
	\begin{itemize}
		\item Function~$f$ in WGAN is usually called $\textit{critic}$.
		\item If parameters $\bphi$ lie in a compact set $\boldsymbol{\Phi} \in [-c, c]^d$ then $f(\bx, \bphi)$ will be $K$-Lipschitz continuous function. 
	\end{itemize}
	\begin{multline*}
		K \cdot W(\pi || p) = \max_{\| f \|_L \leq K} \left[ \bbE_{\pi(\bx)} f(\bx)  - \bbE_{p(\bx)} f(\bx)\right] \geq \\  \geq \max_{\bphi \in \boldsymbol{\Phi}} \left[ \bbE_{\pi(\bx)} f(\bx, \bphi)  - \bbE_{p(\bx)} f(\bx, \bphi )\right]
	\end{multline*}
	\textit{"Weight clipping is a clearly terrible way to enforce a Lipschitz constraint"}
	\myfootnotewithlink{https://arxiv.org/abs/1701.07875}{Arjovsky M., Chintala S., Bottou L. Wasserstein GAN, 2017}
\end{frame}
%=======
\begin{frame}{Outline}
	\tableofcontents
\end{frame}
%=======
\section{Lipschitzness of Wasserstein GAN critic}
%=======
\subsection{WGAN with Gradient Penalty}
%=======
\begin{frame}{Wasserstein GAN with Gradient Penalty}
	\vspace{-0.2cm}
	\begin{figure}
		\centering
		\includegraphics[width=0.9\linewidth]{figs/wgan_gp_weights}
	\end{figure}
	\vspace{-0.2cm} 
	
	\begin{block}{Weight clipping analysis}
		\begin{itemize}
			\item The gradients either grow or decay exponentially.
			\item Gradient penalty makes the gradients more stable.
		\end{itemize}
	\end{block}
	\myfootnotewithlink{https://arxiv.org/abs/1704.00028}{Gulrajani I. et al. Improved Training of Wasserstein GANs, 2017}
\end{frame}
%=======
\begin{frame}{Wasserstein GAN with Gradient Penalty}
	\begin{block}{Theorem}
		Let $\pi(\bx)$ and $p(\bx)$ be two distribution in $\cX$, a compact metric space. Let $\gamma$ be the optimal transportation plan between $\pi(\bx)$ and $p(\bx)$. Then
		\begin{enumerate}
			\item there is 1-Lipschitz function $f^*$ which is the optimal solution of 
			\[
				\max_{\| f \|_L \leq 1} \left[ \bbE_{\pi(\bx)} f(\bx)  - \bbE_{p(\bx)} f(\bx)\right].
			\]
			\item if $f^*$ is differentiable, $\gamma(\by = \bz) = 0$ and $\hat{\bx}_t = t \by + (1 - t) \bz$ with $\by \sim \pi(\bx)$, $\bz \sim p(\bx | \btheta)$, $t \in [0, 1]$ it holds that
			\[
				\bbP_{(\by, \bz) \sim \gamma} \left[ \nabla f^*(\hat{\bx}_t) = \frac{\bz - \hat{\bx}_t}{\| \bz - \hat{\bx}_t \|} \right] = 1.
			\]
		\end{enumerate}
	\end{block}
	\vspace{-0.5cm}
	\begin{block}{Corollary}
		$f^*$ has gradient norm 1 almost everywhere under $\pi(\bx)$ and $p(\bx)$.
	\end{block}

	\myfootnotewithlink{https://arxiv.org/abs/1704.00028}{Gulrajani I. et al. Improved Training of Wasserstein GANs, 2017}
\end{frame}
%=======
\begin{frame}{Wasserstein GAN with Gradient Penalty}
	A differentiable function is 1-Lipschtiz if and only if it has gradients with norm at most 1 everywhere.
	\begin{block}{Gradient penalty}
		\vspace{-0.3cm}
		\[
			W(\pi || p) = \underbrace{\bbE_{\pi(\bx)} f(\bx)  - \bbE_{p(\bx)} f(\bx)}_{\text{original critic loss}} + \lambda \underbrace{\bbE_{U[0, 1]} \left[ \left( \| \nabla f(\hat{\bx}) \|_2 - 1 \right) ^ 2\right]}_{\text{gradient penalty}},
		\]
		\vspace{-0.3cm}
	\end{block}
	\begin{itemize}
		\item Samples $\hat{\bx}_t = t \by + (1 - t) \bz$ with $t \in [0, 1]$ are uniformly sampled along straight lines between pairs of points: $\by$ from the data distribution $\pi(\bx)$ and $\bz$ from the generator distribution $p(\bx | \btheta)$.
		\item Enforcing the unit gradient norm constraint everywhere is intractable, it turns out to be sifficient to enforce it only along these straight lines.
	\end{itemize}

	\myfootnotewithlink{https://arxiv.org/abs/1704.00028}{Gulrajani I. et al. Improved Training of Wasserstein GANs, 2017}
\end{frame}
%=======
\subsection{Spectral Normalization GAN}
%=======
\begin{frame}{Spectral Normalization GAN}
	\begin{block}{Definition}
		$\| \bA \|_2$ is a \textit{spectral norm} of matrix $\bA$:
		\[
			\| \bA \|_2 = \max_{\bh \neq 0} \frac{\|\bA \bh\|_2}{\|\bh\|_2} = \max_{\|\bh\|_2 \leq 1} \| \bA \bh \|_2 = \sqrt{\lambda_{\text{max}}(\bA^T \bA)},
		\]
		where $\lambda_{\text{max}}(\bA^T \bA)$ is the largest eigenvalue value of $\bA^T \bA$.
	\end{block}
	\begin{block}{Statement 1}
		if $\mathbf{g}$ is a K-Lipschitz vector function then 
		\[
			\| \mathbf{g} \|_L \leq K = \sup_\bx \| \nabla \mathbf{g}(\bx) \|_2.
		\]
		\vspace{-0.7cm}
	\end{block}
	\begin{block}{Statement 2}
		Lipschitz norm of superposition is bounded above by product of Lipschitz norms
		\vspace{-0.2cm}
		\[
			\| \mathbf{g}_1 \circ \mathbf{g}_2 \|_L \leq \| \mathbf{g}_1 \|_L \cdot \| \mathbf{g}_2\|_L
		\]
	\end{block}
	\myfootnotewithlink{https://arxiv.org/abs/1802.05957}{Miyato T. et al. Spectral Normalization for Generative Adversarial Networks, 2018}
\end{frame}
%=======
\begin{frame}{Spectral Normalization GAN}
	Let consider the critic $f(\bx, \bphi)$ of the following form:
	\[
		f(\bx, \bphi) = \bW_{K+1} \sigma_K (\bW_K \sigma_{K-1}(\dots \sigma_1(\bW_1 \bx) \dots)).
	\]
	This feedforward network is a superposition of simple functions.
	\begin{itemize}
		\item $\sigma_k$ is a pointwise nonlinearities. We assume that $\| \sigma_k \|_L = 1$ (it holds for ReLU).
		\item $\mathbf{g}(\bx) = \bW^T \bx$ is a linear transformation ($\nabla \mathbf{g}(\bx) = \bW$).
		\[
			\| \mathbf{g} \|_L \leq \sup_\bx \| \nabla \mathbf{g}(\bx)\|_2 = \| \bW \|_2.
		\]
	\end{itemize}
	\vspace{-0.4cm}
	\begin{block}{Critic spectral norm}
		\vspace{-0.4cm}
		\[
			\| f \|_L \leq \| \bW_{K+1}\|_2 \cdot \prod_{k=1}^K  \| \sigma_k \|_L \cdot \| \bW_k \|_2 = \prod_{k=1}^{K+1} \|\bW_k\|_2.
		\]
		\vspace{-0.2cm}
	\end{block}
	If we replace the weights in the critic $f(\bx, \bphi)$ by $\bW^{SN}_k = \bW_k / \|\bW_k\|_2$, we will get $\| f\|_L \leq 1.$ \\
	
	\myfootnotewithlink{https://arxiv.org/abs/1802.05957}{Miyato T. et al. Spectral Normalization for Generative Adversarial Networks, 2018}
\end{frame}
%=======
\begin{frame}{Spectral Normalization GAN}
	How to compute $ \| \bW \|_2 = \sqrt{\lambda_{\text{max}}(\bW^T \bW)}$? \\
	We are not able to apply SVD at each iteration.
	 \begin{block}{Power iteration (PI) method}
	 	\vspace{-0.2cm}
	 	\begin{itemize}
	 		\item $\bu_0$ -- random vector.
	 		\item for $m = 0, \dots, M - 1$: ($M$ is a fixed number of steps)
	 		\vspace{-0.3cm}
	 		\[
	 			\bv_{m+1} = \frac{\bW^T \bu_{m}}{\| \bW^T \bu_{m} \|}, \quad \bu_{m+1} = \frac{\bW \bv_{m+1}}{\| \bW \bv_{m+1} \|}.
	 		\]
	 		\item approximate the spectral norm
	 		\vspace{-0.3cm}
	 		\[
	 			\| \bW \|_2 = \sqrt{\lambda_{\text{max}}(\bW^T \bW)} \approx \bu_{M}^T \bW \bv_{M}.
	 		\]
	 	\end{itemize}
	 \end{block}
	 \vspace{-0.5cm}
 	\begin{block}{SNGAN gradient update}
 		\begin{itemize}
 			\item Apply PI method to get approximation of spectral norm.
 			\item Normalize weights $\bW^{SN}_k = \bW_k / \|\bW_k\|_2$.
 			\item Apply gradient rule to $\bW$.
 		\end{itemize}
 	\end{block}

	\myfootnotewithlink{https://arxiv.org/abs/1802.05957}{Miyato T. et al. Spectral Normalization for Generative Adversarial Networks, 2018}
\end{frame}
%=======
\section{f-divergence minimization}
%=======
\begin{frame}{Divergences}
	\begin{itemize}
		\item Forward KL divergence in maximum likelihood estimation.
		\item Reverse KL in variational inference.
		\item JS divergence in standard GAN.
		\item Wasserstein distance in WGAN.
	\end{itemize}
	\begin{block}{What is a divergence?}
		Let $\cS$ be the set of all possible probability distributions. Then $D: \cS \times \cS \rightarrow \bbR$ is a divergence if 
		\begin{itemize}
			\item $D(\pi || p) \geq 0$ for all $\pi, p \in \cS$;
			\item $D(\pi || p) = 0$ if and only if $\pi \equiv p$.
		\end{itemize}
	\end{block}
	\begin{block}{General divergence minimization task}
		\vspace{-0.3cm}
		\[
			\min_p D(\pi || p)
		\]
		\vspace{-0.7cm}
	\end{block}
	\begin{block}{Chalenge}
		We do not know the real distribution $\pi(\bx)$!
	\end{block}
\end{frame}
%=======
\begin{frame}{f-divergence family}
	
	\begin{block}{f-divergence}
		\vspace{-0.3cm}
		\[
		D_f(\pi || p) = \bbE_{p(\bx)}  f\left( \frac{\pi(\bx)}{p(\bx)} \right)  = \int p(\bx) f\left( \frac{\pi(\bx)}{p(\bx)} \right) d \bx.
		\]
		Here $f: \bbR_+ \rightarrow \bbR$ is a convex, lower semicontinuous function satisfying $f(1) = 0$.
	\end{block}
	\begin{figure}
		\centering
		\includegraphics[width=\linewidth]{figs/f_divs}
	\end{figure}
	\myfootnotewithlink{https://arxiv.org/abs/1606.00709}{Nowozin S., Cseke B., Tomioka R. f-GAN: Training Generative Neural Samplers using Variational Divergence Minimization, 2016}
\end{frame}
%=======
\begin{frame}{f-divergence family}
	\vspace{-0.2cm}
	\begin{block}{Fenchel conjugate}
		\vspace{-0.7cm}
		\[
		f^*(t) = \sup_{u \in \text{dom}_f} \left( ut - f(u) \right), \quad f(u) = \sup_{t \in \text{dom}_{f^*}} \left( ut - f^*(t) \right)
		\]
		\vspace{-0.5cm}
	\end{block}
	\textbf{Important property:} $ f^{**} = f$ for convex $f$.
	\begin{block}{f-divergence}
		\vspace{-0.8cm}
		\begin{multline*}
			D_f(\pi || p) = \bbE_{p(\bx)}  f\left( \frac{\pi(\bx)}{p(\bx)} \right)  = \int p(\bx) {\color{violet}f\left( \frac{\pi(\bx)}{p(\bx)} \right) } d \bx = \\ = \int p(\bx) {\color{violet} \sup_{t \in \text{dom}_{f^*}} \left( \frac{\pi(\bx)}{p(\bx)} t - f^*(t) \right)} d \bx = \\ 
			= \int \sup_{t \in \text{dom}_{f^*}} \left( \pi(\bx)t - p(\bx) f^*(t) \right) d \bx .
		\end{multline*}
		\vspace{-0.6cm}
	\end{block}
	Here we seek value of $t$, which gives us maximum value of $ \pi(\bx)t - p(\bx) f^*(t)$, for each data point $\bx$.
	\myfootnotewithlink{https://arxiv.org/abs/1606.00709}{Nowozin S., Cseke B., Tomioka R. f-GAN: Training Generative Neural Samplers using Variational Divergence Minimization, 2016}
\end{frame}
%=======
\begin{frame}{f-divergence family}
	\vspace{-0.4cm}
	\begin{block}{f-divergence}
		\vspace{-0.3cm}
		\[
		D_f(\pi || p) = \bbE_{p(\bx)}  f\left( \frac{\pi(\bx)}{p(\bx)} \right)  = \int p(\bx) f\left( \frac{\pi(\bx)}{p(\bx)} \right) d \bx.
		\]
		\vspace{-0.4cm}
	\end{block}
	\begin{block}{Variational f-divergence estimation}
		\vspace{-0.8cm}
		\begin{multline*}
			D_f(\pi || p)  = {\color{violet}\int} {\color{teal} \sup_{t \in \text{dom}_{f^*}}} \left( \pi(\bx)t - p(\bx) f^*(t) \right) d \bx \geq \\
			 \geq {\color{teal}\sup_{T \in \cT}} {\color{violet}\int} \left( \pi(\bx)T(\bx) - p(\bx) f^*(T(\bx)) \right) d \bx = \\
			 = \sup_{T \in \cT} \left[\bbE_{\pi}T(\bx) -  \bbE_p f^*(T(\bx)) \right]
		\end{multline*}
	\vspace{-0.6cm}
	\end{block}
	This is a lower bound because of Jensen inequality and restricted class of functions $\cT: \cX \rightarrow \bbR$. 
	
	\myfootnotewithlink{https://arxiv.org/abs/1606.00709}{Nowozin S., Cseke B., Tomioka R. f-GAN: Training Generative Neural Samplers using Variational Divergence Minimization, 2016}
\end{frame}
%=======
\begin{frame}{f-divergence family}
	\begin{block}{Variational divergence estimation}
		\[
			D_f(\pi || p) \geq \sup_{T \in \cT} \left[\bbE_{\pi}T(\bx) -  \bbE_p f^*(T(\bx)) \right]
		\]
		The lower bound is tight for $T^*(\bx) = f'\left( \frac{\pi(\bx)}{p(\bx)} \right)$.
	\end{block}
	\begin{block}{Example (JSD)}
		\begin{itemize}
			\item Let define function $f$ and its conjugate $f^*$
			\[ 
				f(u) = u \log u - (u + 1) \log (u + 1), \quad f^*(t) = - \log (1 - e^t).
			\]
			\item Let reparametrize $T(\bx) = \log D(\bx)$.
		\end{itemize}
		\vspace{-0.4cm}
	\end{block}
	\[
		\min_{G} \max_D \left[ \bbE_{\pi(\bx)} \log D(\bx) + \bbE_{p(\bz)} \log (1 - D(G(\bz))) \right]
	\]

	\myfootnotewithlink{https://arxiv.org/abs/1606.00709}{Nowozin S., Cseke B., Tomioka R. f-GAN: Training Generative Neural Samplers using Variational Divergence Minimization, 2016}
\end{frame}
%=======
\begin{frame}{f-divergence family}
	\begin{block}{Variational divergence estimation}
		\[
			D_f(\pi || p) \geq \sup_{T \in \cT} \left[\bbE_{\pi}T(\bx) -  \bbE_p f^*(T(\bx)) \right]
		\]
	\end{block}
	\textbf{Note:} To evaluate lower bound we only need samples from $\pi(\bx)$ and $p(\bx)$. Hence, we could fit implicit generative model.
	\begin{figure}
		\centering
		\includegraphics[width=1.0\linewidth]{figs/f_div_results}
	\end{figure}

	\myfootnotewithlink{https://arxiv.org/abs/1606.00709}{Nowozin S., Cseke B., Tomioka R. f-GAN: Training Generative Neural Samplers using Variational Divergence Minimization, 2016}
\end{frame}
%=======
\section{Evaluation of likelihood-free models}
%=======
\begin{frame}{Evaluation of likelihood-free models}
	How to evaluate generative models?
	\begin{block}{Likelihood-based models}
		\begin{itemize}
			\item Split data to train/val/test.
			\item Fit model on the train part.
			\item Tune hyperparameters on the validation part.
			\item Evaluate generalization by reporting likelihoods on the test set.
		\end{itemize}
	\end{block}
	\begin{block}{Not all models have tractable likelihoods}
		\begin{itemize}
			\item VAE: compare ELBO values.
			\item GAN: ???
		\end{itemize}
	\end{block}
\end{frame}
%=======
\begin{frame}{Evaluation of likelihood-free models}
	Let take some pretrained image classification model to get the conditional label distribution $p(y | \bx)$ (e.g. ImageNet classifier).
	\begin{block}{What do we want from samples?}
		\begin{itemize}
			\item \textbf{Sharpness}
			\begin{figure}
				\centering
				\includegraphics[width=0.9\linewidth]{figs/sharpness}
			\end{figure}
			The conditional distribution $p(y | \bx)$ should have low entropy (each image $\bx$ should have distinctly recognizable object).
			\item \textbf{Diversity}
			\begin{figure}
				\centering
				\includegraphics[width=0.9\linewidth]{figs/diversity}
			\end{figure}
			The marginal distribution $p(y) = \int p(y | \bx) p(\bx) d \bx$ should have high entropy (there should be as many classes generated as possible).
		\end{itemize}
	\end{block}
	\myfootnotewithlink{https://deepgenerativemodels.github.io}{image credit: https://deepgenerativemodels.github.io}
\end{frame}
%=======
\begin{frame}{Evaluation of likelihood-free models}
	\begin{block}{What do we want from samples?}
		\begin{itemize}
			\item \textbf{Sharpness.}
			The conditional distribution $p(y | \bx)$ should have low entropy (each image $\bx$ should have distinctly recognizable object).
			\item \textbf{Diversity.}
			The marginal distribution $p(y) = \int p(y | \bx) p(\bx) d \bx$ should have high entropy (there should be as many classes generated as possible).
		\end{itemize}
	\end{block}
	\begin{figure}
		\centering
		\includegraphics[width=1.0\linewidth]{figs/is_toy}
	\end{figure}
	\myfootnotewithlink{https://medium.com/octavian-ai/a-simple-explanation-of-the-inception-score-372dff6a8c7a}{image credit: https://medium.com/octavian-ai/a-simple-explanation-of-the-inception-score-372dff6a8c7a}
\end{frame}
%=======
\begin{frame}{Summary}
	\begin{itemize} 
		\item Weight clipping is a terrible way to enforce Lipschitzness. Gradient Penalty adds regularizer to loss that uses neccessary conditions for optimal critic.
		\vfill
		\item Spectral normalization is a weight normalization technique to enforce Lipshitzness, which is helpful for generator and critic.
		\vfill
		\item f-divergence family is a unified framework for divergence minimization, which uses variational approximation. Standard GAN is a special case of it.
		\vfill
		\item We need measure of quality for implicit models like GANs. One way is to analyze sharpness and diversity of samples.
	\end{itemize}
\end{frame}
%=======
\end{document} 