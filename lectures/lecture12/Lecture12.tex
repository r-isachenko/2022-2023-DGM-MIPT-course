\input{../utils/preamble}
\createdgmtitle{12}
%--------------------------------------------------------------------------------
\begin{document}
%--------------------------------------------------------------------------------
\begin{frame}[noframenumbering,plain]
%\thispagestyle{empty}
\titlepage
\end{frame}
%=======
\begin{frame}{Recap of previous lecture}
	Let take some pretrained image classification model to get the conditional label distribution $p(y | \bx)$ (e.g. ImageNet classifier).
	\begin{block}{Evaluation of likelihood-free models}
		\begin{itemize}
			\item Sharpness $\Rightarrow$ low $H(y | \bx) = - \sum_{y} \int_{\bx} p(y, \bx) \log p(y | \bx) d\bx$.
			\item Diversity $\Rightarrow$ high $H(y)  = - \sum_{y} p(y) \log p(y)$.
		\end{itemize}
	\end{block}
	\vspace{-0.3cm}
	\begin{block}{Inception Score}
		\vspace{-0.2cm}
		\[
			IS = \exp(H(y) - H(y | \bx)) = \exp \left( \bbE_{\bx} KL(p(y | \bx) || p(y)) \right)
		\]
		\vspace{-0.5cm}
	\end{block}
	\begin{block}{Frechet Inception Distance}
		\vspace{-0.3cm}
		\[
			D^2 (\pi, p) = \| \mathbf{m}_{\pi} - \mathbf{m}_{p}\|_2^2 + \text{Tr} \left( \bSigma_{\pi} + \bSigma_p - 2 \sqrt{\bSigma_{\pi} \bSigma_p} \right).
		\]
		\vspace{-0.5cm}
	\end{block}
	FID is related to moment matching.
	
	\myfootnote{\href{https://arxiv.org/abs/1606.03498}{Salimans T. et al. Improved Techniques for Training GANs, 2016} \\
	\href{https://arxiv.org/abs/1706.08500}{Heusel M. et al. GANs Trained by a Two Time-Scale Update Rule Converge to a Local Nash Equilibrium, 2017} }
\end{frame}
%=======
\begin{frame}{Recap of previous lecture}
		\begin{itemize}
			\item $\cS_{\pi} = \{\bx_i\}_{i=1}^{n} \sim \pi(\bx)$ -- real samples;
			\item $\cS_{p} = \{\bx_i\}_{i=1}^{n} \sim p(\bx | \btheta)$ -- generated samples.
		\end{itemize}
		Embed samples using pretrained classifier network (as previously):
		\[
			\cG_{\pi} = \{\mathbf{g}_i\}_{i=1}^n, \quad \cG_{p} = \{\mathbf{g}_i\}_{i=1}^n.
		\]
		Define binary function:
		\[
			f(\mathbf{g}, \cG) = 
			\begin{cases}
				1, \text{if exists } \mathbf{g}' \in \cG: \| \mathbf{g}  - \mathbf{g}'\|_2 \leq \| \mathbf{g}' - \text{NN}_k(\mathbf{g}', \cG)\|_2; \\
				0, \text{otherwise.}
			\end{cases}
		\]
		\[
			\text{Precision} (\cG_{\pi}, \cG_{p}) = \frac{1}{n} \sum_{\mathbf{g} \in \cG_{p}} f(\mathbf{g}, \cG_{\pi}); \quad \text{Recall} (\cG_{\pi}, \cG_{p}) = \frac{1}{n} \sum_{\mathbf{g} \in \cG_{\pi}} f(\mathbf{g}, \cG_{p}).
		\]
		\vspace{-0.4cm}
		\begin{figure}
			\includegraphics[width=0.7\linewidth]{figs/pr_k_nearest}
		\end{figure}
		\myfootnotewithlink{https://arxiv.org/abs/1904.06991}{Kynkäänniemi T. et al. Improved precision and recall metric for assessing generative models, 2019}
\end{frame}
%=======
\begin{frame}{Recap of previous lecture}
	\begin{block}{Discrete VAE latents}
		\begin{itemize}
			\item Define dictionary (word book) space $\{\be_k\}_{k=1}^K$, where $\be_k \in \bbR^C$, $K$ is the size of the dictionary.
			\item Our variational posterior $q(c | \bx, \bphi) = \text{Categorical}(\bpi(\bx, \bphi))$ (encoder) outputs discrete probabilities vector.
			\item We sample $c^*$ from $q(c | \bx, \bphi)$ (reparametrization trick analogue).
			\item Our generative distribution $p(\bx | \be_{c^*}, \btheta)$ (decoder).
		\end{itemize}
	\end{block}
	\vspace{-0.3cm}
	\begin{block}{ELBO}
		\vspace{-0.7cm}
		\[
		\mathcal{L} (\bphi, \btheta)  = \mathbb{E}_{q(c | \bx, \bphi)} \log p(\bx | c, \btheta) - KL(q(c| \bx, \bphi) || p(c)) \rightarrow \max_{\bphi, \btheta}.
		\]
		\vspace{-0.7cm}
	\end{block}
	\begin{block}{KL term}
		\vspace{-0.4cm}
		\[
		KL(q(c| \bx, \bphi) || p(c)) = - H(q(c | \bx, \bphi)) + \log K. 
		\]
	\end{block}
	Is it possible to make reparametrization trick? (we sample from discrete distribution now!).
\end{frame}
%=======
\begin{frame}{Outline}
	\tableofcontents
\end{frame}
%=======
\section{Discrete VAE latent representations}
%=======
\subsection{Vector quantization}
%=======
\begin{frame}{Vector quantization}
	\begin{block}{Quantized representation}
		$\bz_q \in \bbR^{C}$  for $\bz \in \bbR^C$ is defined by a nearest neighbor look-up using the shared dictionary space
		\vspace{-0.3cm}
		\[
		\bz_q = \be_{k^*}, \quad \text{where } k^* = \argmin_k \| \bz - \be_k \|.
		\] 
		\vspace{-0.7cm}
	\end{block}
	\begin{itemize}
		\item Let our encoder outputs continuous representation $\bz$. 
		\item Quantization will give us the discrete distribution $q(c | \bx, \bphi)$.
	\end{itemize}
	\vspace{-0.2cm}
	\begin{block}{Quantization procedure}
		If we have tensor with the spatial dimensions we apply the quantization for each of $W \times H$ locations.
		\begin{minipage}[t]{0.65\columnwidth}
			\begin{figure}
				\includegraphics[width=0.8\linewidth]{figs/fqgan_cnn.png}
			\end{figure}
		\end{minipage}%
		\begin{minipage}[t]{0.35\columnwidth}
			\begin{figure}
				\includegraphics[width=0.7\linewidth]{figs/fqgan_lookup}
			\end{figure}
		\end{minipage}
	\end{block}
	\myfootnotewithlink{https://arxiv.org/abs/2004.02088}{Zhao Y. et al. Feature Quantization Improves GAN Training, 2020} 
\end{frame}
%=======
\begin{frame}{Vector Quantized VAE (VQ-VAE)}
	Let VAE latent variable $\bc \in \{1, \dots, K\}^{W \times H}$ is the discrete with spatial-independent variational posterior and prior distributions  
	\vspace{-0.3cm}
	\[
	q(\bc | \bx, \bphi) = \prod_{i=1}^W \prod_{j=1}^H q(c_{ij} | \bx, \bphi); \quad p(\bc) = \prod_{i=1}^W \prod_{j=1}^H \text{Uniform}\{1, \dots, K\}.
	\]
	Let $\bz_e = \text{NN}_e(\bx, \bphi) \in \bbR^{W \times H \times C}$ is the encoder output.
	\begin{block}{Deterministic variational posterior}
		\vspace{-0.6cm}
		\[
		q(c_{ij} = k^* | \bx, \bphi) = \begin{cases}
			1 , \quad \text{for } k^* = \argmin_k \| [\bz_e]_{ij} - \be_k \|; \\
			0, \quad \text{otherwise}.
		\end{cases}
		\]
		$KL(q(c| \bx, \bphi) || p(c))$ term in ELBO is constant, entropy of the posterior is zero.
		\[
		KL(q(c | \bx, \bphi) || p(c)) = - H(q(c | \bx, \bphi)) + \log K = \log K. 
		\]
	\end{block}
	
	\myfootnotewithlink{https://arxiv.org/abs/1711.00937}{Oord A., Vinyals O., Kavukcuoglu K. Neural Discrete Representation Learning, 2017} 
\end{frame}
%=======
\begin{frame}{Vector Quantized VAE (VQ-VAE)}
	\begin{block}{ELBO}
		\vspace{-0.6cm}
		\[
		\mathcal{L} (\bphi, \btheta)  = \mathbb{E}_{q(c | \bx, \bphi)} \log p(\bx | \be_{c} , \btheta) - \log K =  \log p(\bx | \bz_q, \btheta) - \log K,
		\]
		where $\bz_q = \be_{k^*}$, $k^* = \argmin_k \| \bz_e - \be_k \|$.
	\end{block}
	\begin{figure}
		\centering
		\includegraphics[width=0.85\linewidth]{figs/vqvae}
	\end{figure}
	\textbf{Problem:} $\argmin$ is not differentiable.
	\begin{block}{Straight-through gradient estimation}
		\vspace{-0.6cm}
		\[
		\frac{\partial \log p(\bx | \bz_q , \btheta)}{\partial \bphi} = \frac{\partial \log p(\bx | \bz_q, \btheta)}{\partial \bz_q} \cdot {\color{red}\frac{\partial \bz_q}{\partial \bphi}} \approx \frac{\partial \log p(\bx | \bz_q, \btheta)}{\partial \bz_q} \cdot \frac{\partial \bz_e}{\partial \bphi}
		\]
	\end{block}
	\myfootnotewithlink{https://arxiv.org/abs/1711.00937}{Oord A., Vinyals O., Kavukcuoglu K. Neural Discrete Representation Learning, 2017} 
\end{frame}
%=======
\begin{frame}{Vector Quantized VAE-2  (VQ-VAE-2)}
	\begin{block}{Samples 1024x1024}
		\vspace{-0.2cm}
		\begin{figure}
			\centering
			\includegraphics[width=0.63\linewidth]{figs/vqvae2_faces}
		\end{figure}
	\end{block}
	\vspace{-0.6cm}
	\begin{block}{Samples diversity}
		\vspace{-0.2cm}
		\begin{figure}
			\centering
			\includegraphics[width=0.65\linewidth]{figs/vqvae2_diversity}
		\end{figure}
	\end{block}
	\myfootnotewithlink{https://arxiv.org/abs/1906.00446}{Razavi A., Oord A., Vinyals O. Generating Diverse High-Fidelity Images with VQ-VAE-2, 2019} 
\end{frame}
%=======
\subsection{Gumbel-softmax}
%=======
\begin{frame}{Gumbel-softmax trick}
	\begin{itemize}
		\item VQ-VAE has deterministic variational posterior (it allows to get rid of discrete sampling and reparametrization trick).
		\item There is no uncertainty in the encoder output. 
	\end{itemize}
	\vspace{-0.2cm}
	\begin{block}{Gumbel-max trick}
		Let $g_k \sim \text{Gumbel}(0, 1)$ for $k = 1, \dots, K$, i.e. $g = - \log (- \log u)$, $u \sim \text{Uniform}[0, 1]$. Then a discrete random variable
		\vspace{-0.2cm}
		\[
			c = \argmax_k [\log \pi_k + g_k],
		\]
		\vspace{-0.5cm} \\
		has a categorical distribution $c \sim \text{Categorical}(\bpi)$.
	\end{block}
	\begin{itemize}
		\item Let our encoder $q(c | \bx, \bphi) = \text{Categorical}(\bpi(\bx, \bphi))$ outputs logits of $\bpi(\bx, \bphi)$.
		\item We could sample from the discrete distribution using Gumbel-max reparametrization.
	\end{itemize}

	\myfootnote{
	\href{https://arxiv.org/abs/1611.00712}{Maddison C. J., Mnih A., Teh Y. W. The Concrete distribution: A continuous relaxation of discrete random variables, 2016} \\
	\href{https://arxiv.org/abs/1611.01144}{Jang E., Gu S., Poole B. Categorical reparameterization with Gumbel-Softmax, 2016}
	}
\end{frame}
%=======
\begin{frame}{Gumbel-softmax trick}
	\begin{block}{Reparametrization trick (LOTUS)}
		\vspace{-0.7cm}
		\[
			\nabla_{\bphi} \mathbb{E}_{q(c | \bx, \bphi)} \log p(\bx | \be_{c} , \btheta) = \bbE_{\text{Gumbel}(0, 1)} \nabla_{\bphi} \log p(\bx | \be_{k^*} , \btheta),
		\]
		where $k^* = \argmax_k [\log q(k | \bx, \bphi) + g_k]$.
	\end{block}
	\textbf{Problem:} We still have non-differentiable $\argmax$ operation.
	
	\begin{block}{Gumbel-softmax relaxation}
		{\color{violet}Con}{\color{teal}crete} distribution = {\color{violet}\textbf{con}tinuous} + {\color{teal}dis\textbf{crete}}
		\vspace{-0.2cm}
		\[
			\hat{c}_k = \frac{\exp \left(\frac{\log q(k | \bx, \bphi) + g_k}{\tau}\right)}{\sum_{j=1}^K \exp \left(\frac{\log q(j | \bx, \bphi) + g_j}{\tau}\right)}, \quad k = 1, \dots, K.
		\]
		\vspace{-0.4cm} \\
		Here $\tau$ is a temperature parameter. Now we have differentiable operation, but the gradient estimate is biased now.
 	\end{block}
	\myfootnote{
	\href{https://arxiv.org/abs/1611.00712}{Maddison C. J., Mnih A., Teh Y. W. The Concrete distribution: A continuous relaxation of discrete random variables, 2016} \\
	\href{https://arxiv.org/abs/1611.01144}{Jang E., Gu S., Poole B. Categorical reparameterization with Gumbel-Softmax, 2016}
	}
\end{frame}
%=======
\begin{frame}{Gumbel-softmax trick}
	 \vspace{-0.3cm}
	\begin{block}{Concrete distribution}
	 	\begin{figure}
	 		\includegraphics[width=0.8\linewidth]{figs/gumbel_softmax}
	 	\end{figure}
	 	\vspace{-0.7cm}
	 	\begin{figure}
	 		\includegraphics[width=0.8\linewidth]{figs/simplex}
	 	\end{figure} 
	 	\vspace{-0.5cm}
	 \end{block}
	\begin{block}{Reparametrization trick}
		\vspace{-0.4cm}
		\[
			\nabla_{\bphi} \mathbb{E}_{q(c | \bx, \bphi)} \log p(\bx | \be_{c} , \btheta) = \bbE_{\text{Gumbel}(0, 1)} \nabla_{\bphi} \log p(\bx | \bz , \btheta),
		\]
		where $\bz = \sum_{k=1}^K\hat{c}_k \be_k$ (all operations are differentiable now).
	\end{block}
 	\vspace{-0.2cm}
	\myfootnote{
	\href{https://arxiv.org/abs/1611.00712}{Maddison C. J., Mnih A., Teh Y. W. The Concrete distribution: A continuous relaxation of discrete random variables, 2016} \\
	\href{https://arxiv.org/abs/1611.01144}{Jang E., Gu S., Poole B. Categorical reparameterization with Gumbel-Softmax, 2016}
	}
\end{frame}
%=======
\begin{frame}{DALL-E/dVAE}
	\begin{block}{Deterministic VQ-VAE posterior}
		\vspace{-0.3cm}
		\[
			q(\hat{z}_{ij} = k^* | \bx) = \begin{cases}
				1 , \quad \text{for } k^* = \argmin_k \| [\bz_e]_{ij} - \be_k \| \\
				0, \quad \text{otherwise}.
			\end{cases}
		\]
		\vspace{-0.3cm}
	\end{block}
	\begin{itemize}
		\item Gumbel-Softmax trick allows to make true categorical distribution and sample from it.
		\item Since latent space is discrete we could train autoregressive transformers in it.
		\item It is a natural way to incorporate text and image token spaces.
	\end{itemize}
	\begin{figure}
		\includegraphics[width=\linewidth]{figs/dalle}
	\end{figure}
	\myfootnotewithlink{https://arxiv.org/abs/2102.1209}{Ramesh A. et al. Zero-shot text-to-image generation, 2021}
\end{frame}
%=======
\section{Neural ODE}
%=======
\begin{frame}{Neural ODE}
	Consider Ordinary Differential Equation (ODE)
	\begin{align*}
	    \frac{d \bz(t)}{dt} &= f(\bz(t), t, \btheta); \quad \text{with initial condition }\bz(t_0) = \bz_0. \\
	    \bz(t_1) &= \int^{t_1}_{t_0} f(\bz(t), t, \btheta) d t  + \bz_0 = \text{ODESolve}(\bz(t_0), f, t_0,t_1, \btheta).
	\end{align*}
	\vspace{-0.4cm}
	\begin{block}{Euler update step}
		\vspace{-0.6cm}
		\[
		    \frac{\bz(t + \Delta t) - \bz(t)}{\Delta t} = f(\bz(t), t, \btheta) \,\, \Rightarrow \,\, \bz(t + \Delta t) = \bz(t) + \Delta t \cdot f(\bz(t), t, \btheta)
		\]
		\vspace{-0.7cm}
	\end{block}
	\begin{block}{Residual block}
		\begin{minipage}[t]{0.7\columnwidth}
			\vspace{-0.4cm}
			\[
				\bz_{t + 1} = \bz_t + f(\bz_t, \btheta)
			\]
			\vspace{-0.6cm}
			\begin{itemize}
				 \item It is equavalent to Euler update step for solving ODE with $\Delta t = 1$!
				 \item Euler update step is unstable and trivial. There are more sophisticated methods.
			\end{itemize}
		\end{minipage}%
		\begin{minipage}[t]{0.3\columnwidth}
			\vspace{-0.2cm}
			\begin{figure}
			    \centering
			    \includegraphics[width=\linewidth]{figs/resnet_1.png}
			\end{figure}
		\end{minipage}
		\vspace{-0.4cm}
	\end{block}

	\myfootnotewithlink{https://arxiv.org/abs/1806.07366}{Chen R. T. Q. et al. Neural Ordinary Differential Equations, 2018}   
\end{frame}
%=======
\begin{frame}{Neural ODE}
	\begin{block}{Residual block}
	\vspace{-0.4cm}
	\[
	    \bz_{t+1} = \bz_t + f(\bz_t, \btheta).
	\]
	\vspace{-0.4cm}
	\end{block}
	In the limit of adding more layers and taking smaller steps, we parameterize the continuous dynamics of hidden units using an ODE specified by a neural network: 
	\[
	    \frac{d \bz(t)}{dt} = f(\bz(t), t, \btheta); \quad \bz(t_0) = \bx; \quad \bz(t_1) = \by.
	\]
	\begin{minipage}[t]{0.4\columnwidth}
		\begin{figure}
			\centering
			\includegraphics[width=0.8\linewidth]{figs/euler}
		\end{figure}
	\end{minipage}%
	\begin{minipage}[t]{0.6\columnwidth}
		\vspace{-0.4cm}
		\begin{figure}
			\centering
			\includegraphics[width=0.9\linewidth]{figs/resnet_vs_neural_ode.png}
		\end{figure}
	\end{minipage}

	\myfootnotewithlink{https://arxiv.org/abs/1806.07366}{Chen R. T. Q. et al. Neural Ordinary Differential Equations, 2018}   
\end{frame}
%=======
\begin{frame}{Neural ODE}
	\begin{block}{Forward pass (loss function)}
		\vspace{-0.8cm}
		\begin{align*}
			L(\by) = L(\bz(t_1)) &= L\left( \bz(t_0) + \int_{t_0}^{t_1} f(\bz(t), t, \btheta) dt \right) \\ &= L\bigl(\text{ODESolve}(\bz(t_0), f, t_0,t_1, \btheta) \bigr)
		\end{align*}
	\vspace{-0.5cm}
	\end{block}
	\textbf{Note:} ODESolve could be any method (Euler step, Runge-Kutta methods).
	\begin{block}{Backward pass (gradients computation)}
		For fitting parameters we need gradients:
		\[
			\ba_{\bz}(t) = \frac{\partial L(\by)}{\partial \bz(t)}; \quad \ba_{\btheta}(t) = \frac{\partial L(\by)}{\partial \btheta(t)}.
		\]
		In theory of optimal control these functions called \textbf{adjoint} functions. 
		They show how the gradient of the loss depends on the hidden state~$\bz(t)$ and parameters $\btheta$.
	\end{block}

	\myfootnotewithlink{https://arxiv.org/abs/1806.07366}{Chen R. T. Q. et al. Neural Ordinary Differential Equations, 2018}     
\end{frame}
%=======
\begin{frame}{Summary}
	\begin{itemize}
		\item Vector Quantization is the way to create VAE with discrete latent space and deterministic variational posterior. 
		\vfill
		\item Straight-through gradient ignores quantize operation in backprop.
		\vfill
		\item Gumbel-softmax trick relaxes discrete problem to continuous one using Gumbel-max reparametrization trick.
		\vfill
		\item It becomes more and more popular to use discrete latent spaces in the fields of image/video/music generation.
		\vfill		
		\item Residual networks could be interpreted as solution of ODE with Euler method.
	\end{itemize}
\end{frame}
%=======
\end{document} 